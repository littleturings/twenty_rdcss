\documentclass{article}
\usepackage[UTF8]{ctex}
\usepackage{siunitx}
\usepackage{hyperref}

\title{2020 RDCSS: Python \&基础算法结业考核}
\author{施想\\詹有丘}

\hypersetup{
colorlinks=true,
linkcolor=black,
urlcolor=black,
pdftitle={2020 RDCSS: Python\&基础算法结业考核}
}

\begin{document}

\maketitle

\begin{table}[h!]
\centering
\begin{tabular}{cccccc}
题号 & 题目名 & 程序名 & 时间限制 & 内存限制 & 语言限制\\
\hline
1 & \hyperlink{section.1}{罗马数字} & roman & $\SI{1}{s}$ & $\SI{256}{MB}$ & Python\\
2 & \hyperlink{section.2}{牢房} & jail & $\SI{1}{s}$ & $\SI{256}{MB}$ & Python\\
3 & \hyperlink{section.3}{全排列} & perm & $\SI{1}{s}$ & $\SI{256}{MB}$ & Python\\
\end{tabular}
\end{table}

环境: 64位Windows下的Python 3.8.4.

\section{罗马数字 (roman)}
\label{sec:roman}

施想喜欢研究数学史.
这天, 他正在研究罗马数字.

罗马数字包含以下七种字符: I, V, X, L, C, D, M.

\begin{table}[h!]
\centering
\begin{tabular}{cc}
字符 & 数值\\
\hline
I & 1\\
V & 5\\
X & 10\\
L & 50\\
C & 100\\
D & 500\\
M & 1000
\end{tabular}
\end{table}

例如, 罗马数字2写做II, 即为两个并列的1.
12写做XII, 即为X + II. 
27写做XXVII, 即为XX + V + II.

通常情况下, 罗马数字中小的数字在大的数字的右边.
但也存在特例, 例如4不写做IIII, 而是IV.
数字1在数字5的左边, 所表示的数等于大数5减小数1得到的数值4.

同样地, 数字9表示为IX. 这个特殊的规则只适用于以下六种情况:
\begin{itemize}
\item I可以放在V (5) 和X (10) 的左边, 来表示4和9.
\item X可以放在L (50) 和C (100) 的左边, 来表示40和90.
\item C可以放在D (500) 和M (1000) 的左边, 来表示400和900.
\end{itemize}

施想觉得罗马数字非常有趣, 但是罗马数字实在是太难读了!
于是他想要写一个程序来\textbf{计算出给定的罗马数字代表的数值}.

\subsection*{程序名:}

\texttt{roman}

\subsection*{输入 (\texttt{roman.in}):}

输入共一行, 一个字符串. 保证是一个合法的罗马数字, 且其相应的整数值不超过3999.

\subsection*{样例输入1:}

\begin{verbatim}
III
\end{verbatim}

\subsection*{样例输入2:}

\begin{verbatim}
MCMXCIV
\end{verbatim}

\subsection*{输出 (\texttt{roman.out}):}

输出共一行, 即转换成的整数结果.

\subsection*{样例输出1:}

\begin{verbatim}
3
\end{verbatim}

\subsection*{样例输出2:}

\begin{verbatim}
1994
\end{verbatim}

\subsection*{样例输出2解释:}

M = 1000, CM = 900, XC = 90, IV = 4.

\section{牢房 (jail)}
\label{sec:jail}

詹有丘开了一家监狱, 专门接待那些不好好上课的犯人.
为了惩罚犯人, 他每天都会指挥犯人离开自己的房间或者进入自己的房间.

监狱中总共有8间牢房, 排成一排, 每间牢房不是有人住就是空着.

詹有丘的想法非常奇怪, 他制定了一个指挥牢房内犯人进出的规则:

\textbf{如果一间牢房的两个相邻的房间都被占用或都是空的, 那么第二天该牢房就会被占用.
否则, 它就会被空置.}

请注意, 由于监狱中的牢房排成一行, 所以行中的第一个和最后一个房间无法有两个相邻的房间.

詹有丘制定了规则之后, 对于$N$天后监狱的状况非常好奇, 于是他想要写一个程序来预测这一点.
请你来帮帮他.

\subsection*{程序名:}

\texttt{jail}

\subsection*{输入 (\texttt{jail.in}):}

输入共两行.

第一行是8个用空格隔开的整数$c_j$, 用于表示首日监狱的状况.
其中$c_j=0,1$描述了第$j$个牢房的状况, 用$0$表示空牢房, 用$1$表示被占用的牢房.

第二行是一个整数$N$, 满足$1\le N\le 10^9$.

\subsection*{样例输入1:}

\begin{verbatim}
0 1 0 1 1 0 0 1
7
\end{verbatim}

\subsection*{样例输入2:}

\begin{verbatim}
1 0 0 1 0 0 1 0
1000000000
\end{verbatim}

\subsection*{输出 (\texttt{jail.out}):}

输出共一行, 包含8个用空格隔开的整数$c'_j$, 用于表示$N$天后监狱的状况.
其中$c'_j=0,1$描述了第$j$个牢房的状况, 用$0$表示空牢房, 用$1$表示被占用的牢房.

\subsection*{样例输出1:}

\begin{verbatim}
0 0 1 1 0 0 0 0
\end{verbatim}

\subsection*{样例输出1解释:}

下表概述了监狱每天的状况:
\begin{table}[h!]
\centering
\begin{tabular}{c|c}
0 & \texttt{0 1 0 1 1 0 0 1}\\
1 & \texttt{0 1 1 0 0 0 0 0}\\
2 & \texttt{0 0 0 0 1 1 1 0}\\
3 & \texttt{0 1 1 0 0 1 0 0}\\
4 & \texttt{0 0 0 0 0 1 0 0}\\
5 & \texttt{0 1 1 1 0 1 0 0}\\
6 & \texttt{0 0 1 0 1 1 0 0}\\
7 & \texttt{0 0 1 1 0 0 0 0}\\
\end{tabular}
\end{table}

\subsection*{样例输出2:}

\begin{verbatim}
0 0 1 1 1 1 1 0
\end{verbatim}

\subsection*{来源:}

\href{https://leetcode-cn.com/problems/prison-cells-after-n-days}{LeetCode 957.}

\section{全排列 (perm)}
\label{sec:perm}

胡姝婧是一个喜欢有秩序的事物的同学.
她有一个爱好, 就是喜欢摆弄她的$N$个乐高小人.

因为她喜欢秩序, 所以她总是会把她的乐高小人排成从左到右的一行, 还给小人标记了编号.
她每天都会改变乐高小人的队伍, 发现她时常能排出一种她以前从来没排过的顺序.
于是她想要计算一下这$N$个乐高小人到底有哪几种可能的\textbf{不重复的}排队方式.

请你帮助胡姝婧写出程序.
直觉告诉她, \textbf{这可能要用到最近学习过的递归}.
另外, 因为胡姝婧热爱秩序, 请将所有的排列方式\textbf{按字典序升序输出}.

\subsection*{程序名:}

\texttt{perm}

\subsection*{输入 (\texttt{perm.in}):}

输入共一行. 包含一个整数$N$, 满足$1\le N\le10$.

\subsection*{样例输入:}

\begin{verbatim}
3
\end{verbatim}

\subsection*{输出 (\texttt{perm.out}):}

输出共$N!$行, 其中第$j$行是$N$个用空格隔开的整数, 表示第$j$种可能的排列.

\subsection*{样例输出:}

\begin{verbatim}
0 1 2
0 2 1
1 0 2
1 2 0
2 0 1
2 1 0
\end{verbatim}

\end{document}
